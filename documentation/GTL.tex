\documentclass[11pt]{article}
\usepackage{geometry}                % See geometry.pdf to learn the layout options. There are lots.
\geometry{a4paper}                   % ... or a4paper or a5paper or ... 
%\geometry{landscape}                % Activate for for rotated page geometry
%\usepackage[parfill]{parskip}    % Activate to begin paragraphs with an empty line rather than an indent
\usepackage{graphicx}
\usepackage{amssymb}
\usepackage{epstopdf}
\usepackage{listings}
\usepackage{longtable}
\usepackage{colortbl}
\usepackage[table]{xcolor}
\usepackage{underscore}
\usepackage{tikzpagenodes}
\usetikzlibrary{calc}
\usepackage{eso-pic}

\AddToShipoutPictureBG{%
\begin{tikzpicture}[overlay,remember picture]
\draw[line width=0.5pt]
      ($ (current page text area.north west) $)
      rectangle
      ($ (current page text area.south east) $);
\end{tikzpicture}
}

\makeatletter
\newcommand*{\pmzeroslash}{%
  \nfss@text{%
    \sbox0{0}%
    \sbox2{/}%
    \sbox4{%
      \raise\dimexpr((\ht0-\dp0)-(\ht2-\dp2))/2\relax\copy2 %
    }%
    \ooalign{%
      \hfill\copy4 \hfill\cr
      \hfill0\hfill\cr
    }%
    \vphantom{0\copy4 }% correct overall height and depth of the symbol
  }%
}
\makeatother



\definecolor{seagreen}{RGB}{64,100,120}
\definecolor{darkpink}{RGB}{200,100,120}
\definecolor{light-gray}{gray}{0.85}
\definecolor{medium-gray}{gray}{0.5}

\DeclareGraphicsRule{.tif}{png}{.png}{`convert #1 `dirname #1`/`basename #1 .tif`.png}

\lstdefinelanguage{goilTemplate}
{
  morekeywords= {
	  after,
	  before,
	  between,
	  default,
	  do,
	  down,
	  else,
	  elsif,
	  emptylist,
	  emptymap,
	  end,
	  error,
	  exists,
	  false,
	  for,
	  foreach,
	  from,
	  here,
	  let,
	  loop,
	  no,
	  if,
	  in,
	  mod,
	  not,
	  or,
	  step,
	  template,
	  then,
	  to,
	  true,
	  yes,
	  warning,
	  while,
	  write,
	  !,
	  ?
	}
}

\lstset{
  language=goilTemplate,
  emph={
    variable,
    expression,
    string,
    instruction_list,
    template_file_name,
    getter,
    setter,
    hierarchy
  },
  emphstyle=\em,
  moredelim=[s][\color{blue}]{\%}{\%},
  moredelim=[s]{"}{"},
  morecomment=[l][\color{medium-gray}\itshape]{\#},
  basicstyle=\ttfamily\small,
  morekeywords=\bfseries
}

\newcommand{\character}[1]{{\small\ttfamily `{#1}'}}

\newcommand{\argu}[1]{{\ttfamily #1}}
\newcommand{\ctype}[1]{{\small\ttfamily #1}}
\newcommand{\cdata}[1]{{\ttfamily #1}}
\newcommand{\var}[1]{{\ttfamily\em #1}}
\newcommand{\va}{\small\ttfamily\em}
\newcommand{\cfunction}[1]{{\ttfamily #1}}
\newcommand{\cmacro}[1]{{\small\ttfamily #1}}
\newcommand{\constant}[1]{{\small\ttfamily #1}}
\newcommand{\oilobj}[1]{{\small\ttfamily #1}}
\newcommand{\oilattr}[1]{{\small\ttfamily #1}}
\newcommand{\oilval}[1]{{\small\ttfamily #1}}
\newcommand{\member}[1]{{\small\ttfamily #1}}
\newcommand{\mem}{\small\ttfamily}
\newcommand{\tool}[1]{{\em #1}}

\title{GTL Template language}
\author{Jean-Luc B\'echennec}
%\date{}                                           % Activate to display a given date or no date

\begin{document}
\maketitle
%\section{}
%\subsection{}

\tableofcontents

\section{Data types}

GTL supports the following data types:
\begin{description}
\item[int] arbitrary precision integer numbers. The GMP library is used;
\item[float] 64 bits floating point numbers;
\item[bool] standard boolean;
\item[type] the type of a data;
\item[string] unicode strings;
\item[struct] structured data;
\item[list] lists of data, may be accessed as a table;
\item[map] map (aka dictionary) of data;
\item[unconstructed] an unconstructed variable.
\end{description}

Each type has its set of operators, getters and setters. The expression, for getters, or the variable, for setters, is called the {\em target}. Getters return a value related to the data but do not change it. They are used to get information from a data or to convert it into another type. Setter may target a literal expressions. Setters change the data and do not return anything. Setter may only target a variable. Getters and setters may have arguments. Syntax for getters without argument is as follow:

\begin{lstlisting}[language=goilTemplate]
[expression getter]
\end{lstlisting}

When the getter takes arguments, they are listed after a colon and separated commas as follow:

\begin{lstlisting}[language=goilTemplate]
[expression getter : arg1, arg2, ..., argN]
\end{lstlisting}

Syntax for setters without arguments is as follow:

\begin{lstlisting}[language=goilTemplate]
[!variable setter]
\end{lstlisting}

When the setter takes arguments, they are listed after a colon and separated by commas as follow:

\begin{lstlisting}[language=goilTemplate]
[!variable setter : arg1, arg2, ..., argN]
\end{lstlisting}

\subsection{getters applicable to any data type}

\paragraph{\lstinline{type}} returns the data type of the expression. See the section \ref{sec:type}.

\paragraph{\lstinline{isANumber}} returns \lstinline{true} if the expression is a number: \lstinline{int} or \lstinline{float}, \lstinline{false} otherwise.

\subsection{\lstinline{int} data type}

The \lstinline{int} data type support arbitrary precision arithmetic by using the GNU Multiple Precision Arithmetic Library (GMP).

\subsubsection{\lstinline{int} operators}

The \lstinline{int} datatype supports the following operators:

\paragraph{Unary operators}~

\rowcolors{1}{white}{light-gray}
\begin{longtable}{>{\ttfamily}l|>{\ttfamily}l|p{3.34in}}
{\bf Operator}&{\bf Expression type}&{\bf Meaning}\\
\hline\endhead
 {+}&
  {int $\leftarrow$ +int}&
  {Plus operator. No effect}\\
 {-}&
  {int $\leftarrow$ -int}&
  {Minus operator. Negation}\\
 {\raisebox{-1.2mm}{\textasciitilde}}&
  {int $\leftarrow$ \raisebox{-1.2mm}{\textasciitilde}int}&
  {Not operator. Complementation by 1}\\
\end{longtable}

\paragraph{Binary arithmetic operators}~

\rowcolors{1}{white}{light-gray}
\begin{longtable}{>{\ttfamily}l|>{\ttfamily}l|p{3.14in}}
{\bf Operator}&{\bf Expression type}&{\bf Meaning}\\
\hline\endhead
 {+}&
  {int $\leftarrow$ int + int}&
  {Addition}\\
 {-}&
  {int $\leftarrow$ int - int}&
  {Substraction}\\
 {*}&
  {int $\leftarrow$ int * int}&
  {Multiplication}\\
 {/}&
  {int $\leftarrow$ int / int}&
  {Division}\\
 {mod}&
  {int $\leftarrow$ int mod int}&
  {Modulus}\\
\end{longtable}

\paragraph{Binary bitwise operators}~

\rowcolors{1}{white}{light-gray}
\begin{longtable}{>{\ttfamily}l|>{\ttfamily}l|p{3.22in}}
{\bf Operator}&{\bf Expression type}&{\bf Meaning}\\
\hline\endhead
 {\&}&
  {int $\leftarrow$ int \& int}&
  {bitwise and}\\
 {|}&
  {int $\leftarrow$ int \rule{1pt}{1.5ex} int}&
  {bitwise or}\\
 {\^~}&
  {int $\leftarrow$ int \^{} int}&
  {bitwise exclusive or}\\
 {<<}&
  {int $\leftarrow$ int << int}&
  {shift left}\\
 {>>}&
  {int $\leftarrow$ int >> int}&
  {shift right}\\
\end{longtable}

\paragraph{Comparison operators}~

\rowcolors{1}{white}{light-gray}
\begin{longtable}{>{\ttfamily}l|>{\ttfamily}l|p{3.15in}}
{\bf Operator}&{\bf Expression type}&{\bf Meaning}\\
\hline\endhead
 {!=}&
  {bool $\leftarrow$ int != int}&
  {Not equal}\\
 {==}&
  {bool $\leftarrow$ int == int}&
  {Equal}\\
 {>}&
  {bool $\leftarrow$ int > int}&
  {Greater than}\\
 {<}&
  {bool $\leftarrow$ int < int}&
  {Lower than}\\
 {>=}&
  {bool $\leftarrow$ int >= int}&
  {Greater or equal}\\
 {<=}&
  {bool $\leftarrow$ int <= int}&
  {Lower or equal}\\
\end{longtable}

\subsubsection{\lstinline{int} getters}

\rowcolors{1}{white}{light-gray}
\begin{longtable}{>{\ttfamily}l|l|p{3.13in}}
{\bf getter}&{\bf Type}&{\bf Meaning}\\
\hline\endhead
 {string}&
  {string}&
  {Returns a string decimal representation of the int expression. \texttt{[42 string]} returns string \texttt{"42"}.}\\
 {hexString}&
  {string}&
  {Returns a string hexadecimal representation of the int expression prefixed by \texttt{0x}. If the expression is negative a `-' is inserted before. \texttt{[42 hexString]} returns string \texttt{"0x2A"}. \texttt{[-1 hexString]} returns string \texttt{"-0x1"}.}\\
 {xString}&
  {string}&
  {Returns a string hexadecimal representation of the int expression. If the expression is negative a `-' is inserted before. \texttt{[42 xString]} returns string \texttt{"2A"}. \texttt{[-42 xString]} returns string \texttt{"-2A"}.}\\
 {numberOfBytes}&
  {int}&
  {Returns the number of bytes needed to store an unsigned expression. \texttt{[255 numberOfBytes]} returns \texttt{1}, \texttt{[256 numberOfBytes]} returns \texttt{2}}\\
 {signedNumberOfBytes}&
  {int}&
  {Returns the number of bytes needed to store a signed expression. \texttt{[127 numberOfBytes]} returns \texttt{1}, \texttt{[128 numberOfBytes]} returns \texttt{2}}\\
 {numberOfBits}&
  {int}&
  {Returns the number of bits needed to store an unsigned expression. \texttt{[63 numberOfBits]} returns \texttt{6}, \texttt{[64 numberOfBits]} returns \texttt{7}}\\
 {signedNumberOfBits}&
  {int}&
  {Returns the number of bits needed to store a signed expression. \texttt{[63 signedNumberOfBits]} returns \texttt{7}, \texttt{[64 signedNumberOfBits]} returns \texttt{8}}\\
 {sign}&
  {int}&
  {Returns \texttt{-1} if the expression is strictly negative, \texttt{0} if it is null and \texttt{+1} if the expression is strictly positive.}\\
 {fitsUnsignedInByte}&
  {bool}&
  {Returns \texttt{true} if the expression fits in an unsigned byte, \texttt{false} otherwise.}\\
 {fitsSignedInByte}&
  {bool}&
  {Returns \texttt{true} if the expression fits in a signed byte, \texttt{false} otherwise.}\\
 {fitsUnsignedInWord}&
  {bool}&
  {Returns \texttt{true} if the expression fits in an unsigned 16 bits word, \texttt{false} otherwise.}\\
 {fitsSignedInWord}&
  {bool}&
  {Returns \texttt{true} if the expression fits in a signed 16 bits word, \texttt{false} otherwise.}\\
 {fitsUnsignedInLong}&
  {bool}&
  {Returns \texttt{true} if the expression fits in an unsigned 32 bits long, \texttt{false} otherwise.}\\
 {fitsSignedInLong}&
  {bool}&
  {Returns \texttt{true} if the expression fits in a signed 32 bits long, \texttt{false} otherwise.}\\
 {fitsUnsignedInLongLong}&
  {bool}&
  {Returns \texttt{true} if the expression fits in an unsigned 64 bits long long, \texttt{false} otherwise.}\\
 {fitsSignedInLongLong}&
  {bool}&
  {Returns \texttt{true} if the expression fits in a signed 64 bits long long, \texttt{false} otherwise.}\\
 {abs}&
  {int}&
  {Returns the absolute value of the expression.}\\
 {bitAtIndex}&
  {bool}&
  {This getter takes one argument: \texttt{index}. It returns \texttt{true} if the bit at index \texttt{index} is set and \texttt{false} otherwise. \texttt{index} 0 corresponds to the lowest significant bit. \texttt{[1 bitAtIndex: 0]} returns \texttt{true}}\\
\end{longtable}

\subsubsection{\lstinline{int} setters}

\rowcolors{1}{white}{light-gray}
\begin{longtable}{>{\ttfamily}l|p{3.84in}}
{\bf setter}&{\bf Meaning}\\
\hline\endhead
 {setBitAtIndex}&
  {This setter takes two arguments. The first one, \texttt{value}, is a bool. The second one, \texttt{index}, is the index of the bit to set. if \texttt{value} is \texttt{true} the bit is set to \texttt{1} and to \texttt{0} otherwise. Assuming \texttt{a} contains \texttt{0} at start, \texttt{[!a setBitAtIndex: true, 0]} sets \texttt{a} to \texttt{1}.}\\
 {complementBitAtIndex}&
  {This setter takes one argument, \texttt{index}, which is the index of the bit to complement.  Assuming \texttt{a} contains \texttt{1} at start, \texttt{[!a complementBitAtIndex: 1]} sets \texttt{a} to \texttt{3}.}\\
\end{longtable}

\subsection{The \lstinline{float} data type}

The \lstinline{float} data type is the standard IEEE784 64 bits floating point number.

\subsubsection{\lstinline{float} operators}

The \lstinline{float} data type supports the following operators:

\paragraph{Unary operators}~

\rowcolors{1}{white}{light-gray}
\begin{longtable}{>{\ttfamily}l|>{\ttfamily}l|p{3.34in}}
{\bf Operator}&{\bf Expression type}&{\bf Meaning}\\
\hline\endhead
 {+}&
  {float $\leftarrow$ +float}&
  {Plus operator. No effect}\\
 {-}&
  {float $\leftarrow$ -float}&
  {Minus operator. Negation}\\
\end{longtable}

\paragraph{Binary arithmetic operators}~

\rowcolors{1}{white}{light-gray}
\begin{longtable}{>{\ttfamily}l|>{\ttfamily}l|p{2.83in}}
{\bf Operator}&{\bf Expression type}&{\bf Meaning}\\
\hline\endhead
 {+}&
  {float $\leftarrow$ float + float}&
  {Addition}\\
 {-}&
  {float $\leftarrow$ float - float}&
  {Substraction}\\
 {*}&
  {float $\leftarrow$ float * float}&
  {Multiplication}\\
 {/}&
  {float $\leftarrow$ float / float}&
  {Division}\\
\end{longtable}

\paragraph{Comparison operators}~

\rowcolors{1}{white}{light-gray}
\begin{longtable}{>{\ttfamily}l|>{\ttfamily}l|p{2.83in}}
{\bf Operator}&{\bf Expression type}&{\bf Meaning}\\
\hline\endhead
 {!=}&
  {bool $\leftarrow$ float != float}&
  {Not equal}\\
 {==}&
  {bool $\leftarrow$ float == float}&
  {Equal}\\
 {>}&
  {bool $\leftarrow$ float > float}&
  {Greater than}\\
 {<}&
  {bool $\leftarrow$ float < float}&
  {Lower than}\\
 {>=}&
  {bool $\leftarrow$ float >= float}&
  {Greater or equal}\\
 {<=}&
  {bool $\leftarrow$ float <= float}&
  {Lower or equal}\\
\end{longtable}


\subsubsection{\lstinline{float} getters}

\rowcolors{1}{white}{light-gray}
\begin{longtable}{>{\ttfamily}l|l|p{4.16in}}
{\bf getter}&{\bf Type}&{\bf Meaning}\\
\hline\endhead
 {string}&
  {string}&
  {Returns a string representation of the float expression. \texttt{[4.2 string]} returns string \texttt{"4.2"}.}\\
 {cos}&
  {float}&
  {Returns the cosine of a float expression expressed in radian.}\\
 {sin}&
  {float}&
  {Returns the sine of a float expression expressed in radian.}\\
 {tan}&
  {float}&
  {Returns the tangent of a float expression expressed in radian.}\\
 {cosDegree}&
  {float}&
  {Returns the cosine of a float expression expressed in degree.}\\
 {sinDegree}&
  {float}&
  {Returns the sine of a float expression expressed in degree.}\\
 {tanDegree}&
  {float}&
  {Returns the tangent of a float expression expressed in degree.}\\
 {exp}&
  {float}&
  {Returns the exponentiation of a float expression.}\\
 {logn}&
  {float}&
  {Returns the natural logarithm of a float expression.}\\
 {log2}&
  {float}&
  {Returns the logarithm base 2 of a float expression.}\\
 {log10}&
  {float}&
  {Returns the logarithm base 10 of a float expression.}\\
 {sqrt}&
  {float}&
  {Returns the square root of a float expression.}\\
 {power}&
  {float}&
  {This getter takes one argument, \texttt{p}. It returns the expression raised to the  power of \texttt{p}.}\\
\end{longtable}

\subsection{The \texttt{\small string} data type}

The \texttt{string} data type supports unicode. A literal string is delimited by a pair of \texttt{"}. Literal strings support special characters:

\rowcolors{1}{white}{light-gray}
\begin{longtable}{>{\ttfamily}l|>{\ttfamily}l}
{\bf Escape sequence}&{\bf Corresponding character}\\
\hline\endhead
 {\textbackslash f}&
  {form feed}\\
 {\textbackslash n}&
  {new line}\\
 {\textbackslash r}&
  {return}\\
 {\textbackslash t}&
  {horizontal tab}\\
 {\textbackslash v}&
  {vertical tab}\\
 {\textbackslash\textbackslash}&
  {backslash}\\
 {\textbackslash\pmzeroslash}&
  {null character}\\
 {\textbackslash u\textsl{nnnn}}&
  {unicode character with code \textsl{nnnn} in hexadecimal}\\
 {\textbackslash U\textsl{nnnnnnnn}}&
  {unicode character with code \textsl{nnnnnnnn} in hexadecimal}\\
\end{longtable}

\subsubsection{\texttt{\small string} operators}


The \texttt{\small string} data type supports the following operators:

\paragraph{Binary operator}~

\rowcolors{1}{white}{light-gray}
\begin{longtable}{>{\ttfamily}l|>{\ttfamily}l|p{2.6in}}
{\bf Operator}&{\bf Expression type}&{\bf Meaning}\\
\hline\endhead
 {+}&
  {string $\leftarrow$ string + string}&
  {Concatenation}\\
\end{longtable}

\paragraph{Comparison operators}~

\rowcolors{1}{white}{light-gray}
\begin{longtable}{>{\ttfamily}l|>{\ttfamily}l|p{2.67in}}
{\bf Operator}&{\bf Expression type}&{\bf Meaning}\\
\hline\endhead
 {!=}&
  {bool $\leftarrow$ string != string}&
  {Not equal}\\
 {==}&
  {bool $\leftarrow$ string == string}&
  {Equal}\\
 {>}&
  {bool $\leftarrow$ string > string}&
  {Greater than}\\
 {<}&
  {bool $\leftarrow$ string < string}&
  {Lower than}\\
 {>=}&
  {bool $\leftarrow$ string >= string}&
  {Greater or equal}\\
 {<=}&
  {bool $\leftarrow$ string <= string}&
  {Lower or equal}\\
\end{longtable}


\subsubsection{\texttt{\small string} getters}

\rowcolors{1}{white}{light-gray}
\begin{longtable}{>{\ttfamily}l|l|p{2.68in}}
{\bf getter}&{\bf Type}&{\bf Meaning}\\
\hline\endhead
 {HTMLRepresentation}&
  {string}&
  {Returns a representation of the string suitable for an HTML encoded representation. \character{\&} is encoded by \cdata{\&amp;} , \character{"} by \cdata{\&quot;} , \character{<} by \cdata{\&lt;} and \character{>} by \cdata{\&gt;} .}\\
 {identifierRepresentation}&
  {string}&
  {Returns an unique representation of the string conforming to a C identifier. Any Unicode character that is not a latin letter is transformed into its hexadecimal code point value, enclosed by \character{_} characters. This representation is unique: two different strings are transformed into different C identifiers. For example: \cdata{value3} is transformed to \cdata{value_33_}; \cdata{+=} is transformed to \cdata{_2B__3D_};
\cdata{An_Identifier} is transformed to \cdata{An_5F_Identifier}.}\\
 {fileExists}&
  {bool}&
  {Returns \texttt{true} if a file exists at the target path, \texttt{false} otherwise.}\\
 {length}&
  {integer}&
  {Returns the number of characters in the string}\\
 {lowercaseString}&
  {string}&
  {Returns the lowercased representation of the string.}\\
 {capitalized}&
  {string}&
  {if the string is empty, this getter returns the empty string; otherwise, it returns the string with the first character being replaced with the corresponding upper case character.}\\
 {uppercaseString}&
  {string}&
  {Returns uppercased representation of the receiver}\\
 {leftSubString}&
  {string}&
  {Returns the sub-string from the beginning of the target and with the number of characters passed as argument. If the sub-string is longer that the target, the target is returned. \texttt{["Hello\textvisiblespace World\textvisiblespace !" leftSubString : 5]} returns \texttt{"Hello"}.}\\
 {rightSubString}&
  {string}&
  {Returns the sub-string from the end of the target and with the number of characters passed as argument. If the sub-string is longer that the target, the target is returned. \texttt{[["Hello\textvisiblespace World\textvisiblespace !" leftSubString : 11] rightSubString: 5]} returns \texttt{"World"}.}\\
 {subString}&
  {string}&
  {Returns the sub-string from the \texttt{index} passed as first argument and with the number of characters passed as second argument. If the \texttt{index} is out of the target, the empty string is returned. If the number of characters is greater than the sub-string, the sub-string is returned. \texttt{["Hello\textvisiblespace World\textvisiblespace !" subString : 6, 5]} returns \texttt{"World"}. \texttt{["Hello" subString : 10, 3]} returns the empty string. \texttt{["Hello" subString : 2, 10]} returns \texttt{"llo"}.}\\
 {reversedString}&
  {string}&
  {Returns a mirrored string. \texttt{["Hello\textvisiblespace World\textvisiblespace !" reversedString]} returns \texttt{"!\textvisiblespace dlroW\textvisiblespace olleH"}.}\\
 {componentsSeparatedByString}&
  {list}&
  {This getter takes one string argument: \texttt{separator}. The target is cut into pieces according to the separator and a list of the pieces is returned. \texttt{["Hello\textvisiblespace World\textvisiblespace !" componentsSeparatedByString : " "]} returns \texttt{@( "Hello", "World", "!" )}.}\\
 {columnPrefixedBy}&
  {string}&
  {This getter takes one string argument: \texttt{prefix}. Return the target with each line prefixed by \texttt{prefix}. \texttt{["Hello\textbackslash nWorld" columnPrefixedBy : "\#\textvisiblespace"]} returns \texttt{"\#\textvisiblespace Hello\textbackslash n\#\textvisiblespace World"}.}\\
 {wrap}&
  {string}&
  {Wraps the target to a width. This getter takes two int arguments: \texttt{width} and \texttt{shift}. The target is assumed to contain paragraphs separated by \texttt{\textbackslash n}. Returns the target with each paragraph wrapped to \texttt{width}. In addition, each line of the paragraph except the first one is prefixed by \texttt{shift} spaces. \texttt{["Hello\textvisiblespace beautiful\textvisiblespace World.\textbackslash nHow\textvisiblespace are\allowbreak\textvisiblespace\allowbreak you" wrap : 6, 2]} returns \texttt{"Hello\textbackslash n\allowbreak\textvisiblespace\textvisiblespace beautiful\textbackslash n\textvisiblespace\textvisiblespace World\textbackslash nHow\textbackslash n\textvisiblespace\textvisiblespace are\textbackslash n  you"}.}\\
 {subStringExists}&
  {bool}&
  {This getter takes one argument, \texttt{subString}. It returns \texttt{true} if the sub-string \texttt{subString} is found in the target, \texttt{false otherwise}.}\\
 {replaceString}&
  {string}&
  {This getter takes two argument, \texttt{find} and \texttt{replace}. It returns the target where each occurrence of \texttt{find} is replaced by \texttt{replace}.}\\
 {envVar}&
  {string}&
  {Returns the value of the target environment variable. If it does not exists, \texttt{envVar} returns the empty string.}\\
 {envVarExists}&
  {bool}&
  {Returns \texttt{true} if target environment variable exists, \texttt{false} otherwise.}\\
\end{longtable}

\subsection{The \lstinline{bool} data type}

A true literal bool can be written as \texttt{true} or \texttt{yes} and a false literal bool can be written as \texttt{false} or \texttt{no}.

\subsubsection{\lstinline{bool} operators}


The \lstinline{bool} data type supports the following operators:

\paragraph{Unary operator}~

\rowcolors{1}{white}{light-gray}
\begin{longtable}{>{\ttfamily}l|>{\ttfamily}l|p{3.36in}}
{\bf Operator}&{\bf Expression type}&{\bf Meaning}\\
\hline\endhead
 {\raisebox{-1.2mm}{\textasciitilde}}&
  {bool $\leftarrow$ bool}&
  {logical not}\\
\end{longtable}

\paragraph{Binary operator}~

\rowcolors{1}{white}{light-gray}
\begin{longtable}{>{\ttfamily}l|>{\ttfamily}l|p{3.08in}}
{\bf Operator}&{\bf Expression type}&{\bf Meaning}\\
\hline\endhead
 {\&}&
  {bool $\leftarrow$ bool \& bool}&
  {logical and}\\
 {|}&
  {bool $\leftarrow$ bool \rule{1pt}{1.5ex} bool}&
  {logical or}\\
 {\^~}&
  {bool $\leftarrow$ bool \^{} bool}&
  {logical exclusive or}\\
\end{longtable}

\paragraph{Comparison operators}~


\vspace{2mm}
\noindent
For comparison operators, \texttt{false} is considered to be lower than \texttt{true}.

\rowcolors{1}{white}{light-gray}
\begin{longtable}{>{\ttfamily}l|>{\ttfamily}l|p{3in}}
{\bf Operator}&{\bf Expression type}&{\bf Meaning}\\
\hline\endhead
 {!=}&
  {bool $\leftarrow$ bool != bool}&
  {Not equal}\\
 {==}&
  {bool $\leftarrow$ bool == bool}&
  {Equal}\\
 {>}&
  {bool $\leftarrow$ bool > bool}&
  {Greater than}\\
 {<}&
  {bool $\leftarrow$ bool < bool}&
  {Lower than}\\
 {>=}&
  {bool $\leftarrow$ bool >= bool}&
  {Greater or equal}\\
 {<=}&
  {bool $\leftarrow$ bool <= bool}&
  {Lower or equal}\\
\end{longtable}

\subsubsection{\lstinline{bool} getters}

\rowcolors{1}{white}{light-gray}
\begin{longtable}{>{\ttfamily}l|l|p{4.01in}}
{\bf getter}&{\bf Type}&{\bf Meaning}\\
\hline\endhead
 {trueOrFalse}&
  {string}&
  {Returns a string representation, \texttt{"true"} or \texttt{"false"} of the bool expression.}\\
 {string}&
  {string}&
  {Returns a string representation, \texttt{"true"} or \texttt{"false"} of the bool expression.}\\
 {yesOrNo}&
  {string}&
  {Returns a string representation, \texttt{"yes"} or \texttt{"no"} of the bool expression.}\\
 {TRUEOrFALSE}&
  {string}&
  {Returns a string representation, \texttt{"TRUE"} or \texttt{"FALSE"} of the bool expression.}\\
 {YESOrNO}&
  {string}&
  {Returns a string representation, \texttt{"YES"} or \texttt{"NO"} of the bool expression.}\\
 {int}&
  {int}&
  {Returns an int representation, \texttt{1} or \texttt{0} of the bool expression.}\\
\end{longtable}

\subsection{The \lstinline{struct} data type}

The struct data type allows to store a heterogeneous set of data in one variable. Struct members are accessed by using the \var{::} separator. If \var{A} is a struct, \var{A::B} refers to field \var{B} of \var{A}.

A literal struct is defined as follow:

\begin{lstlisting}[language=goilTemplate]
@{ a: 1, b: 2, c: 3 }
\end{lstlisting}

This define a struct with fields a, b and c and respective values 1, 2 and 3.

\subsubsection{\lstinline{struct} operators}

The \lstinline{struct} data type supports the following operators:

\rowcolors{1}{white}{light-gray}
\begin{longtable}{>{\ttfamily}l|>{\ttfamily}l|p{2.682in}}
{\bf Operator}&{\bf Expression type}&{\bf Meaning}\\
\hline\endhead
 {!=}&
  {bool $\leftarrow$ struct != struct}&
  {Not equal}\\
 {==}&
  {bool $\leftarrow$ struct == struct}&
  {Equal}\\
\end{longtable}

Two structs are equal if:
\begin{itemize}
\item they have the same number of field
\item they have the same field names
\item they have the same field values
\end{itemize}

\subsubsection{\lstinline{struct} getter}

\rowcolors{1}{white}{light-gray}
\begin{longtable}{>{\ttfamily}l|l|p{4.43in}}
{\bf getter}&{\bf Type}&{\bf Meaning}\\
\hline\endhead
 {map}&
  {map}&
  {Returns a map representation.}\\
\end{longtable}


\subsection{The \lstinline{list} data type}

The list data type allows to store a list of data in one variable. list members are accessed by using \var{[<number>]} where \var{<number>} is the rank of the element starting at 0. If \var{A} is a list, \var{A[0]} refers to element 0 of \var{A}.

A literal list is defined as follow:

\begin{lstlisting}[language=goilTemplate]
@( 1, 2, 3 )
\end{lstlisting}

This define a list of int with elements 1, 2 and 3.

\subsubsection{\lstinline{list} operators}

The \lstinline{list} data type supports the following operators:

\paragraph{Binary operators}~

\rowcolors{1}{white}{light-gray}
\begin{longtable}{>{\ttfamily}l|>{\ttfamily}l|p{3.08in}}
{\bf Operator}&{\bf Expression type}&{\bf Meaning}\\
\hline\endhead
 {+}&
  {list $\leftarrow$ list + any}&
  {add \texttt{any} add the end of the list}\\
 {|}&
  {list $\leftarrow$ list | list}&
  {Concatenate lists}\\
\end{longtable}

\paragraph{Comparison operators}~

\rowcolors{1}{white}{light-gray}
\begin{longtable}{>{\ttfamily}l|>{\ttfamily}l|p{3in}}
{\bf Operator}&{\bf Expression type}&{\bf Meaning}\\
\hline\endhead
 {!=}&
  {bool $\leftarrow$ list != list}&
  {Not equal}\\
 {==}&
  {bool $\leftarrow$ list == list}&
  {Equal}\\
\end{longtable}

Two structs are equal if:
\begin{itemize}
\item they have the same number of elements
\item they have the same elements values
\end{itemize}

\subsubsection{\lstinline{list} getters}

\rowcolors{1}{white}{light-gray}
\begin{longtable}{>{\ttfamily}l|l|p{4.41in}}
{\bf getter}&{\bf Type}&{\bf Meaning}\\
\hline\endhead
 {length}&
  {int}&
  {Returns the number of elements in the list.}\\
 {first}&
  {any}&
  {Returns the first element of the list.}\\
 {last}&
  {any}&
  {Returns the last element of the list.}\\
 {mapBy}&
  {map}&
  {\cfunction{mapBy} takes on string argument which is the field (for a struct list item) or the key (for a map list item) used as key to store the element in the resulting map. It returns a map where each element is the element of the list with the key being the corresponding field/key.}\\
\end{longtable}

\paragraph{example of \cfunction{mapBy}}~

The following code snippet:

\begin{lstlisting}[language=goilTemplate]
let myList := @(
  @{
    age : 18,
    height : 180,
    name : "Arnold"
  },
  @{
    age : 22,
    height : 170,
    name : "Bob"
  },
  @{
    age : 29,
    height : 175,
    name : "John"
  }
)

let myMap := [myList mapBy : "name"]
display myMap
\end{lstlisting}

outputs:
\begin{lstlisting}[language=goilTemplate]
map: @[
    "Arnold" :>
        struct: @{
            age :>
                integer: 18
            height :>
                integer: 180
            name :>
                string: "Arnold"
        }
    "Bob" :>
        struct: @{
            age :>
                integer: 22
            height :>
                integer: 170
            name :>
                string: "Bob"
        }
    "John" :>
        struct: @{
            age :>
                integer: 29
            height :>
                integer: 175
            name :>
                string: "John"
        }
]
\end{lstlisting}


\end{document}  